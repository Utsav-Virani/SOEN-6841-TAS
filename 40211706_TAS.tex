\documentclass{article}
\usepackage{titlesec}
\usepackage{lipsum}
\usepackage{hyperref}
\usepackage{graphicx}
\usepackage{appendix}
\usepackage[left=1 in,right=1 in,top=1 in,bottom=1 in]{geometry}

% Remove the red boxes around links
\hypersetup{
    colorlinks=true,
    linkcolor=black, % You can change link colors to your preference
    citecolor=blue, % You can change citation colors to your preference
    urlcolor=black % You can change URL colors to your preference
}

\begin{document}
% Title Page
\begin{titlepage}
    \centering
    \includegraphics[width=0.8\textwidth]{image.jpeg}\par % Adjust the width as needed
    % \includegraphics[width=0.8\textwidth]{image2.png}\par % Adjust the width as needed
     \vspace{2cm}
    {\scshape\Large SOEN 6841: TAS REPORT \par}
    \vspace{1.5cm}
    {\scshape\Huge What are the habits of successful project managers?\par}
    \vspace{1.5cm}
    \vspace{1.5cm}
    {\large Advisor: Professor Pankaj Kamthan\par}
    \vspace{1.5cm}
    {\large By: Utsav Virani (Student ID: 40211706)\par}
    \vspace{1cm}
    {\large \today\par}
\end{titlepage}

\tableofcontents

\newpage

\section{Abstract}
The success or failure of astronomy and engineering projects hinges on numerous factors, with one of the most influential being the proficiency of the project manager (PM). This research investigates the key traits, philosophies, and techniques that distinguish skilled and qualified project managers, drawing parallels between successful astronomy and engineering projects\cite{Warner_Mark}.\\

In the realm of astronomy projects, the combined skills and qualifications of the project manager emerge as a pivotal factor influencing project outcomes. This study delves into the common characteristics shared among successful project managers, identifying at least seven key traits that contribute to their effectiveness. These traits encompass skills, habits, and character attributes that play a crucial role in project management. The research explores topics such as scope and quality management, cost and schedule control, project team structures, risk management strategies, stakeholder management, and general project execution.\\

The examination of project management success extends beyond astronomy projects to the broader domain of project implementation. By synthesizing insights from relevant academic literature, industry best practices, and real-world experience \cite{Nieto-Rodriguez2021}, the study sheds light on the nuanced habits that define an effective and efficient project manager. These habits include communicative skills, adept management tactics, planning aptitude, and competency in managing risks.\\

Furthermore, the research builds a foundational understanding of the basic principles of project management before delving deeper into the attributes shared among successful project managers. The evaluation encompasses how these qualities facilitate the realization of key project objectives, including time management, adaptability, team building, and technical competence.\\

By integrating insights from astronomy and engineering projects, this research aims to contribute to a comprehensive understanding of the habits that lead to project management success. The findings provide valuable guidance for aspiring project managers, organizational leaders, and stakeholders involved in the dynamic and challenging arena of project implementation.

\newpage

\section{Introduction}
Project management operates as a dynamic "living system" within the broader "autopoietic organization" of a company, following its intricate internal dynamics to ensure project success. At its core are project managers who, through a unique language comprising technical-operative modality and mindset, navigate both Adaptive and Predictive project lifecycles. As noted by Akbar Azwir, external factors like government decisions can significantly impact projects, necessitating a focus on building strong project resilience \cite{Lorenzo, Daniele}.\\

In this rapidly evolving project management landscape, the traits and behaviors of effective project managers emerge as critical factors influencing project outcomes. This study delves into the essential habits that contribute to project management success, recognizing that organizations must adapt to unprecedented challenges and opportunities. The exploration includes habits such as effective communication, strategic decision-making, planning acumen, risk management proficiency, stakeholder engagement, and overall project execution prowess. By unraveling the intricacies of project management within the context of organizational dynamics, the research aims to shed light on these habits that empower project managers to navigate crises successfully and deliver projects effectively.

\subsection{Motivation}
In the age of disruptive technologies, global connectivity, and changing customer expectations, project management is no longer merely imperative but rather a strategic necessity. This research intends to reveal the habits responsible for the success of projects so that they can be used to guide novice project managers while helping organizations enhance their project management.

\subsection{Problem Statement}
Project management is a key element determining organizational success, but the specific behaviors that typify successful project managers are ambiguous. These habits are not precisely known resulting in possible disagreements and poor leadership that will not offer proper risk avoidance procedures. The importance of clarity on such key habits cannot be overemphasized in a dynamically changing business environment. The purpose of this study is to address the gap in existing knowledge, which is critical for improving current project management methods that are aligned with the changing needs of modern projects. These results will be of much value to the organizations that aim for successful project outcomes and individual project managers.


\subsection{Objective}
This research seeks to analyze and document the habits of successful project managers systematically. This way we hope to bring out the skills and behavior that make them good leaders. This study is designed to assist upcoming and accomplished project managers in acquiring practical lessons that can help formulate their professional growth plan as well as reform project management strategy in the organizations. In essence, the study intends to participate in a larger debate on what makes for effective project management, thereby serving individual professionals, teams, and organizations seeking better outcomes.



\newpage
\section{Background Material}

\subsection{Interpersonal Skills and Communication in Project Management}
Effective communication and interpersonal skills are not just supplementary but central to the success of project management. These skills are crucial for establishing and maintaining positive relationships with team members, stakeholders, and clients, which are fundamental traits for success in this field \cite{Interpersonal_skills}.\\\\
Effective communication involves not only conveying information but also actively listening and understanding the needs and feedback of others. Project managers with strong interpersonal skills can effectively manage complex team dynamics, resolve conflicts, and ensure clear communication, which is vital for project cohesion and success. Emotional intelligence, which encompasses self-awareness, empathy, and social skills, further enhances a project manager's ability to lead effectively \cite{Warner_Mark}.\\

\subsection{Project Leadership and Team Building}
Effective leadership in project management goes beyond mere oversight of tasks. It involves inspiring, guiding, and mentoring team members. Successful project managers understand the strengths of each team member and align them with project objectives.\\
\\This approach not only enhances individual performance but also fosters a sense of belonging and purpose within the team. Leaders who invest in team development create a culture of continuous improvement and learning that is crucial for adapting to changing project demands. Strong team leadership ensures that each member is engaged and contributes to the project's success in a meaningful way\cite{Project_Management}.


\subsection{Adaptability and Problem-Solving}
In today's world, project management is constantly evolving, which means that those who are successful in this field need to have strong problem-solving skills and the ability to adapt quickly to new challenges. This could be due to technological advances, changes in market conditions, or any obstacles that arise within a project. \\\\It's not enough for project managers to simply change their plans; they need to do it efficiently and effectively to limit disruptions and maintain the project's momentum. To solve problems in this context, managers often need to think creatively and come up with innovative solutions that work within the project's constraints.\cite{Project_Management_Leadership}


\subsection{Ethical Decision-Making and Integrity}
Integrity and ethical decision-making are fundamental to the long-term success of projects and the reputation of the managing organization. Project managers often face a variety of ethical dilemmas, ranging from resource allocation to stakeholder engagement and conflict of interest scenarios. \\\\It is crucial for project managers to make decisions that adhere to a strong ethical framework, as this is essential for maintaining stakeholder trust and ensuring project sustainability. Ethical principles and their practical application in diverse project environments have gained increasing attention, underscoring the need for managers to be well-versed in this aspect of project management\cite{Project_Management_Techniques}.


\newpage
\section{Methods \& Methodology}
\subsection{Research Approach}
The study employs a mixed-methods approach, combining both qualitative and quantitative research methodologies to gain a deeper understanding of the habits that distinguish successful project managers. By utilizing statistical analysis and in-depth qualitative insights, the study aims to provide a comprehensive analysis of these habits.
\subsection{Quantitative Analysis}
As part of the research, we will analyze existing studies and surveys related to project management success. The analysis will focus on identifying common traits and habits among successful project managers. Additionally, we will explore the correlations between specific habits and project outcomes. Empirical studies and industry surveys will serve as the sources for this data.\cite{relationship_between_project_manage}
\subsection{Qualitative Analysis}
The study includes a qualitative component involving interviews and case studies. Interviews will be conducted with experienced project managers to gain personal insights into the habits and traits that contribute to their success. \\\\In addition, successful project management instances will be analyzed through case studies to identify common habits and approaches. These narratives will offer a more detailed understanding of the practical applications of these habits in different project contexts\cite{Project_Management}.
\subsection{Ethical Considerations}
In conducting interviews and case studies, we will strictly adhere to ethical considerations such as informed consent, confidentiality, and the right to withdraw. This aligns with the ethical leadership traits crucial for successful project management\cite{Project_Management_Techniques}.
\subsection{Data Analysis}
Both quantitative and qualitative data sources will be analyzed to identify patterns and commonalities in the habits of successful project managers. This analysis will be crucial in developing a comprehensive understanding of the key factors that contribute to project management success. Any spelling, grammar, or punctuation errors have been corrected.


\newpage
\section{Results Obtained}
\subsection{Under what conditions:}
The study revealed that specific circumstances had a significant impact on the efficiency of project managers.
\begin{itemize}
    \item Highly Collaborative Environments: Project managers were most successful in teams where open communication and collaboration were fostered, especially in those with a strong culture of trust and mutual respect.
    \item Dynamic and Changing Project Scenarios: In fast-paced and ever-changing work environments, project managers who possess exceptional adaptability and problem-solving skills are highly valued. These individuals have the ability to quickly and efficiently modify plans and strategies as needed, allowing them to excel and achieve success even in the face of frequent changes. Their capacity to navigate dynamic situations with ease enables them to lead their teams with confidence and produce exceptional outcomes\cite{relationship_between_project_manage}.
    \item Ethical and Transparent Cultures: In organizations with a strong ethical framework and clear communication channels, projects led by managers who prioritize ethical decision-making and transparency tend to have higher success rates. This is because such managers are more likely to foster a culture of trust and accountability, which in turn leads to better collaboration and more effective problem-solving. Additionally, by being transparent about their decision-making processes and involving team members in important decisions, these managers can help to ensure that everyone is aligned and working towards the same goals. Ultimately, this approach can lead to better outcomes for the organization as a whole, as well as increased job satisfaction and engagement among team members. Projects led by managers who prioritize ethical decision-making and transparency tend to have higher success rates, especially in organizations with a strong ethical framework and clear communication channels\cite{Project_Management_Techniques}.
\end{itemize}

\subsection{constrains}

\newpage
\section{Conclusion and Futureworks}
\subsection{Improvements}
\subsection{limitations}
\subsection{Conclusion}



\newpage
\section*{References}
\addcontentsline{toc}{section}{References}
\vspace*{-35pt}
\renewcommand{\refname}{}
\begin{thebibliography}{99}
\bibitem{Nieto-Rodriguez2021}
A. Nieto-Rodriguez and Y. Khelifi, “Is Project Management the Right Career for You?,” Harvard Business Review, Dec. 27, 2021. \url{https://hbr.org/2021/12/is-project-management-the-right-career-for-you}

\bibitem{Brrlrlry}
Brrlrlry - ds.amu.edu.et. Available at: 
\url{https://ds.amu.edu.et/xmlui/bitstream/handle/123456789/10108/5-phase%20project%20management.pdf?sequence=1&isAllowed=y}

\bibitem{Warner_Mark}
Warner, M. and Summers, R. (2016) The seven Habits of Highly Effective Project Managers, SPIE Digital Library. Available at: \url{https://www.spiedigitallibrary.org/conference-proceedings-of-spie/9911/99110M/The-seven-habits-of-highly-effective-project-managers/10.1117/12.2228381.full?SSO=1}. 

\bibitem{Lorenzo_Daniele}
Lorenzo, Daniele. (2021). The Project Manager “Interfaces” in a Crisis Scenario. European Scientific Journal, ESJ. 17. 61. 10.19044/esj.2021.v17n30p61. 


\bibitem{Interpersonal_skills}
G. Levin, Interpersonal Skills for Portfolio, Program, and Project Managers. Oakland Berrett-Koehler Publishers, Incorporated, 2014.


\bibitem{Project_Management}
P. Bhola, Project Leadership and Team Building in Global Project Management. Partridge Publishing, 2017.


\bibitem{Project_Management_Leadership}
R. G. \& N. Lopez, ‘The Link between Project Management Leadership and Project Success’, Dissertation, 2011.


\bibitem{Project_Management_Techniques}
Singh, R. and Jankovitz, L. (2018), "Effective Project Management Techniques to Prepare Information Professionals for the Future Workforce", Project Management in the Library Workplace (Advances in Library Administration and Organization, Vol. 38), Emerald Publishing Limited, Leeds, pp. 279-294.\url{https://doi.org/10.1108/S0732-067120180000038017}


\bibitem{relationship_between_project_manage}
'The relationship between project manager interpersonal skills and information technology project success - ProQuest,' www.proquest.com. https://www.proquest.com/docview/1619581957?pq-origsite=gscholar&fromopenview=true.


\end{thebibliography}

\newpage

\section{Acknowledgements}
I would like to acknowledge the invaluable contributions of ChatGPT, Perplexity, and Research Gate for the research papers and articles And also references to journals from Google Scholar.

\end{document}

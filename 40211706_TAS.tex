\documentclass{article}
\usepackage{titlesec}
\usepackage{lipsum}
\usepackage{hyperref}
\usepackage{graphicx}
\usepackage{appendix}
\usepackage[left=1 in,right=1 in,top=1 in,bottom=1 in]{geometry}

% Remove the red boxes around links
\hypersetup{
    colorlinks=true,
    linkcolor=black, % You can change link colors to your preference
    citecolor=blue, % You can change citation colors to your preference
    urlcolor=black % You can change URL colors to your preference
}

\begin{document}
% Title Page
\begin{titlepage}
    \centering
    \includegraphics[width=0.8\textwidth]{image.jpeg}\par % Adjust the width as needed
     \vspace{2cm}
    {\scshape\Large SOEN 6481: TAS REPORT \par}
    \vspace{1.5cm}
    {\scshape\Huge What are the habits of successful project managers?\par}
    \vspace{1.5cm}
    \vspace{1.5cm}
    {\large Advisor: Professor Pankaj Kamthan\par}
    \vspace{1.5cm}
    {\large By: Utsav Virani (Student ID: 40211706)\par}
    \vspace{1cm}
    {\large \today\par}
\end{titlepage}

\tableofcontents

\newpage

\section{Abstract}
Project management comes first among successful organizations, and the position of project manager determines how a project will end up being successful or otherwise. This study tries to examine what succession looks like for a project manager in the turbulent arena of project implementation. The paper outlines various habits that constitute an effective and efficient project manager drawing from a synthesis of relevant academic literature, industry best practices, and real-world experience \cite{Nieto-Rodriguez2021}.


This study builds the foundation of basic principles of project management before diving deeper into the list of attributes shared among successful project managers. In this respect, they include communicative habits, management tactics, planning aptitude, and competency in managing risks. This article also evaluates these qualities regarding how they facilitate the realization of objectives such as time management, adaptability, building teams, and technical competence.

\newpage

\section{Introduction}
In today’s rapidly evolving project management ecosystem, project managers play a critical role in driving the success and longevity of initiatives spanning diverse fields and industries. This study was undertaken because there is an increased realization that it is the traits and behaviors of effective project managers that determine the project outcomes. As organizations are forced by unprecedented challenges and opportunities to stay competitive and attain their strategic goals, it is high time that they discover the essential habits that promote project management success.

\subsection{Motivation}
In the age of disruptive technologies, global connectivity, and changing customer expectations, project management is no longer merely imperative but rather a strategic necessity. This research intends to reveal the habits responsible for the success of projects so that they can be used to guide novice project managers while helping organizations enhance their project management.

\subsection{Problem Statement}
Project management is a key element determining organizational success, but the specific behaviors that typify successful project managers are ambiguous. These habits are not precisely known resulting in possible disagreements and poor leadership that will not offer proper risk avoidance procedures. The importance of clarity on such key habits cannot be overemphasized in a dynamically changing business environment. The purpose of this study is to address the gap in existing knowledge, which is critical for improving current project management methods that are aligned with the changing needs of modern projects. These results will be of much value to the organizations that aim for successful project outcomes and individual project managers.


\subsection{Objective}
This research seeks to analyze and document the habits of successful project managers systematically. This way we hope to bring out the skills and behavior that make them good leaders. This study is designed to assist upcoming and accomplished project managers in acquiring practical lessons that can help formulate their professional growth plan as well as reform project management strategy in the organizations. In essence, the study intends to participate in a larger debate on what makes for effective project management, thereby serving individual professionals, teams, and organizations seeking better outcomes.

\newpage


\begin{thebibliography}{99}

\bibitem{Nieto-Rodriguez2021}
A. Nieto-Rodriguez and Y. Khelifi, “Is Project Management the Right Career for You?,” Harvard Business Review, Dec. 27, 2021. \url{https://hbr.org/2021/12/is-project-management-the-right-career-for-you}

\bibitem{YourSecondReferenceLabel}
Brrlrlry - ds.amu.edu.et. Available at: 
\url{https://ds.amu.edu.et/xmlui/bitstream/handle/123456789/10108/5-phase%20project%20management.pdf?sequence=1&isAllowed=y}

\end{thebibliography}


\end{document}

\documentclass{article}
\usepackage{titlesec}
\usepackage{lipsum}
\usepackage{hyperref}
\usepackage{graphicx}
\usepackage{appendix}
\usepackage[left=1 in,right=1 in,top=1 in,bottom=1 in]{geometry}

% Remove the red boxes around links
\hypersetup{
    colorlinks=true,
    linkcolor=black, % You can change link colors to your preference
    citecolor=blue, % You can change citation colors to your preference
    urlcolor=black % You can change URL colors to your preference
}

\begin{document}
% Title Page
\begin{titlepage}
    \centering
    \includegraphics[width=0.8\textwidth]{image.jpeg}\par % Adjust the width as needed
    % \includegraphics[width=0.8\textwidth]{image2.png}\par % Adjust the width as needed
     \vspace{2cm}
    {\scshape\Large SOEN 6841: TAS REPORT \par}
    \vspace{1.5cm}
    {\scshape\Huge What are the habits of successful project managers?\par}
    \vspace{1.5cm}
    \vspace{1.5cm}
    {\large Advisor: Professor Pankaj Kamthan\par}
    \vspace{1.5cm}
    {\large By: Utsav Virani (Student ID: 40211706)\par}
    \vspace{1cm}
    {\large \today\par}
\end{titlepage}

\tableofcontents

\newpage

\section{Abstract}
The success or failure of astronomy and engineering projects hinges on numerous factors, with one of the most influential being the proficiency of the project manager (PM). This research investigates the key traits, philosophies, and techniques that distinguish skilled and qualified project managers, drawing parallels between successful astronomy and engineering projects\cite{Warner_Mark}.\\

In the realm of astronomy projects, the combined skills and qualifications of the project manager emerge as a pivotal factor influencing project outcomes. This study delves into the common characteristics shared among successful project managers, identifying at least seven key traits that contribute to their effectiveness. These traits encompass skills, habits, and character attributes that play a crucial role in project management. The research explores topics such as scope and quality management, cost and schedule control, project team structures, risk management strategies, stakeholder management, and general project execution.\\

The examination of project management success extends beyond astronomy projects to the broader domain of project implementation. By synthesizing insights from relevant academic literature, industry best practices, and real-world experience \cite{Nieto-Rodriguez2021}, the study sheds light on the nuanced habits that define an effective and efficient project manager. These habits include communicative skills, adept management tactics, planning aptitude, and competency in managing risks.\\

Furthermore, the research builds a foundational understanding of the basic principles of project management before delving deeper into the attributes shared among successful project managers. The evaluation encompasses how these qualities facilitate the realization of key project objectives, including time management, adaptability, team building, and technical competence.\\

By integrating insights from astronomy and engineering projects, this research aims to contribute to a comprehensive understanding of the habits that lead to project management success. The findings provide valuable guidance for aspiring project managers, organizational leaders, and stakeholders involved in the dynamic and challenging arena of project implementation.

\newpage

\section{Introduction}
Project management operates as a dynamic "living system" within the broader "autopoietic organization" of a company, following its intricate internal dynamics to ensure project success. At its core are project managers who, through a unique language comprising technical-operative modality and mindset, navigate both Adaptive and Predictive project lifecycles. As noted by Akbar Azwir, external factors like government decisions can significantly impact projects, necessitating a focus on building strong project resilience \cite{Lorenzo, Daniele}.\\

In this rapidly evolving project management landscape, the traits and behaviors of effective project managers emerge as critical factors influencing project outcomes. This study delves into the essential habits that contribute to project management success, recognizing that organizations must adapt to unprecedented challenges and opportunities. The exploration includes habits such as effective communication, strategic decision-making, planning acumen, risk management proficiency, stakeholder engagement, and overall project execution prowess. By unraveling the intricacies of project management within the context of organizational dynamics, the research aims to shed light on these habits that empower project managers to navigate crises successfully and deliver projects effectively.

\subsection{Motivation}
In the age of disruptive technologies, global connectivity, and changing customer expectations, project management is no longer merely imperative but rather a strategic necessity. This research intends to reveal the habits responsible for the success of projects so that they can be used to guide novice project managers while helping organizations enhance their project management.

\subsection{Problem Statement}
Project management is a key element determining organizational success, but the specific behaviors that typify successful project managers are ambiguous. These habits are not precisely known resulting in possible disagreements and poor leadership that will not offer proper risk avoidance procedures. The importance of clarity on such key habits cannot be overemphasized in a dynamically changing business environment. The purpose of this study is to address the gap in existing knowledge, which is critical for improving current project management methods that are aligned with the changing needs of modern projects. These results will be of much value to the organizations that aim for successful project outcomes and individual project managers.


\subsection{Objective}
This research seeks to analyze and document the habits of successful project managers systematically. This way we hope to bring out the skills and behavior that make them good leaders. This study is designed to assist upcoming and accomplished project managers in acquiring practical lessons that can help formulate their professional growth plan as well as reform project management strategy in the organizations. In essence, the study intends to participate in a larger debate on what makes for effective project management, thereby serving individual professionals, teams, and organizations seeking better outcomes.



\newpage
\section{Background Material}

\newpage
\section{Methods \& Methodology}
\subsection{How did we approach the problem?}
\subsection{What techniques are used in the analysis of results?}

\newpage
\section{Results Obtained}
\subsection{Under what conditions}
\subsection{constrains}

\newpage
\section{Conclusion and Futureworks}
\subsection{Improvements}
\subsection{limitations}
\subsection{Conclusion}



\newpage
\section*{References}
\addcontentsline{toc}{section}{References}
\vspace*{-35pt}
\renewcommand{\refname}{}
\begin{thebibliography}{99}
\bibitem{Nieto-Rodriguez2021}
A. Nieto-Rodriguez and Y. Khelifi, “Is Project Management the Right Career for You?,” Harvard Business Review, Dec. 27, 2021. \url{https://hbr.org/2021/12/is-project-management-the-right-career-for-you}

\bibitem{Brrlrlry}
Brrlrlry - ds.amu.edu.et. Available at: 
\url{https://ds.amu.edu.et/xmlui/bitstream/handle/123456789/10108/5-phase%20project%20management.pdf?sequence=1&isAllowed=y}

\bibitem{Warner_Mark}
Warner, Mark \& Summers, Richard. (2016). The seven habits of highly effective project managers. 99110M. 10.1117/12.2228381.

\bibitem{Lorenzo_Daniele}
Lorenzo, Daniele. (2021). The Project Manager “Interfaces” in a Crisis Scenario. European Scientific Journal, ESJ. 17. 61. 10.19044/esj.2021.v17n30p61. 

\end{thebibliography}

\newpage

\section{Acknowledgements}
I would like to acknowledge the invaluable contributions of ChatGPT, Perplexity, and Research Gate for the research papers and articles And also references to journals from Google Scholar.

\end{document}
